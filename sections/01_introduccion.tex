\chapter{Introduction}\label{cap:introduccion}
There is little doubt about the importance of the challenge agriculture is facing nowadays: how to be able to supply the whole world increasing population without using up all the available, but limited, resources of our planet.

Optimizing agricultural procedures is absolutely paramount, and technology and innovation play a leading role in achieving a more efficient and sustainable production systems.

In this line of work, and as the aim of this project, we suggest carrying out a system that will daily assist the farmer by implementing agricultural sensors for the monitoring and control of the harvest. Thereby succeeding in creating a more ecological, environmentally friendly and competent way of managing both time and resources.


\section{Green Revolution}

It is interesting to mention the Green Revolution, or the Third Agricultural Revolution. During the 1940s a lot of programs for agricultural development were promoted\cite{greenrevolution1}.

These were commonly referred as the ``green revolution'' since their goal was to develop high-yielding cereal varieties so as to alleviate hunger to poverty. However, it did not turn out as it was expected.

By the 1970s, althought the cereal production had increased enormously, the poverty remained the same. It was concluded by experts and critics that the main reason for this to happen was that the agricultural technology had been adopted mainly by large, commercial farmers instead of the great majority of peasant-farmers.

An obvious question arises: why has is it so difficult to devise technology appropriate for small farmers?\cite{greenrevolution2}.

-----------------------------------------------------------------------------------------------------

There are several explanations for this, still, most of them are not correct whatsoever. There is an interpretation that is worth discussing as it might be partially true. It states: despite Green Revolution heads and scientists were enthusiastic about increasing production, they never really committed to alleviating hunger among the poor.

According to Harwood, there is evidence that suggests that for some participants, this was not the main purpose. Regardless, most of the scientists and planners were truly concerned about reducing poverty rates.

That being the case, what was the reason the green revolution did not work as it should have?

It is not just a matter of technology.

Background conditions that supported the peasant farming were absolutely crucial. 
