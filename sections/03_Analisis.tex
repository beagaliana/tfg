\chapter{Problem analysis and budget implementation}\label{cap:analisis}

\section{Introduction}
In this chapter both hardware and software shall be broken down in accordance to an estimated budget. We will take a detailed look at the computational cost as well as the economic. Moreover, the implemented system and its components will be specified and defined.

\section{Goal of the prototype}
There are some typical aspects of a budget and problem analysis that are not going to be addressed, such as information requirements or functional and non-functional requirements.

This is because of the very nature of the project. As it has been mentioned before, the main focus of the project has been to offer alternatives that guarantees privacy. It is not the aim of the thesis to propose a model that generates economic profit, but to offer the user - the farmer, for this particular case - a more respectful and free alternative to manage their crops and harvest.

It is in our interests, however, that a budget for the implementation of the prototype is estimated so if anyone might be interested in reproducing a model based precicsely on ours, can have a reliable reference on the economic costs.

\section{Budget}
A chart below is attached in order to give a visual estimation of the costs.

\begin{table}[]
\begin{tabular}{llll}
\hline
	\color{purple} 
		     & \textbf{Quantity}& \textbf{Price per unit (euro)}& \textbf{Total} \\ \hline
	\color{lightgrey} 
Arduino ATMega board & 1                                        & 35                                                 & 35                                    \\
FC-28 sensor         & 1                                        & 3                                                  & 3                                     \\
	\color{lightgrey} 
Photoresistor sensor & 1                                        &                                                    &                                       \\
Raspberry Pi 4 2GB   & 1                                        & 43,95                                              & 43,95                                 \\
\color{lightgrey} 
LEDs                 & 5                                        &                                                    &                                      
\end{tabular}
\end{table}

\section{Hourly distribution of workload}
The graph attached below will give a visual distribution of the workload for the realization of this project along the time. The number of total hours invested are 360 with an error margin that will be assumed as negligible.

\begin{table}[]
\begin{tabular}{llll}
\hline
        \color{purple}
                     & \textbf{Hours} \\ \hline
        \color{lightgrey}
Arduino ATMega board & 1           \\
FC-28 sensor         & 1           \\
        \color{lightgrey}
Photoresistor sensor & 1           \\
Raspberry Pi 4 2GB   & 1           \\
\color{lightgrey}
LEDs                 & 5           
\end{tabular}
\end{table}


\section{Conclusions}
One of the main conclusions that can be drawn from this section is that the main issue of this prototype is not merely economic, as might be thought, but focuses in offering a different way of managing a farm (or any other harvesting) that does not compromise the privacy and security of the user.

Because there is no third party company processing the data and because the software used is licensed as GPLv2, it can be assured and guaranteed that the treatment of the data is always with the consent and knowledge of the user and owner.
