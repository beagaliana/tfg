\chapter{Problem analysis and budget implementation}\label{cap:analisis}

\section{Introduction}
In this chapter both hardware and software shall be broken down in accordance to an estimated budget. We will take a detailed look at the computational cost as well as the economic. Moreover, the implemented system and its components will be specified and defined.

\section{Goal of the prototype}
There are some typical aspects of a budget and problem analysis that are not going to be addressed, such as information requirements or functional and non-functional requirements.

This is because of the very nature of the project. As it has been mentioned before, the main focus of the project has been to offer alternatives that guarantees privacy. It is not the aim of the thesis to propose a model that generates economic profit, but to offer the user - the farmer, for this particular case - a more respectful and free alternative to manage their crops and harvest.

It is in our interests, however, that a budget for the implementation of the prototype is estimated so if anyone might be interested in reproducing a model based precicsely on ours, can have a reliable reference on the economic costs.

\section{Hardware budget}
The charts below attached give a visual estimation of the costs.

The hardware elements are categorised into different tables depending on the purpose they have served for the project. For example, a connector USB A to USB C will go into the Raspberry Pi table as it has been used for supplying with power the Raspberry Pi.

All the components appearing in the table are provided with a link to the place where they have been adquired, or if borrowed, the place where they can be found.

Besides the elements contained in the charts, for the realization of this project a laptop or computer, an extra monitor screen and a stable internet connection are required.

\begin{table}[H]
\caption{Arduino implementation}
\begin{tabular}{llll}
\hline
	& \textbf{Quantity}& \textbf{Price per unit (euro)}& \textbf{Total} \\ \hline
	\rowcolor{lightgray}
\href{https://store.arduino.cc/arduino-mega-2560-rev3}{Arduino ATMega board} & 1                                        & 35                                                 & 35                                    \\
\href{https://televoz.org/9496_128fc28_modulo-para-arduino-igrometro---sensor-de-humedad-fc28}{FC-28 sensor}         & 1                                        & 3                                                  & 3                                     \\
	\rowcolor{lightgray}
\href{https://televoz.org/5593_000ldr01_celula-fotoconductora-ldr-01}{LDR sensor}	     & 1                                        &0,30                                                    & 0,30                                      \\
\href{https://televoz.org/4859_104bp001_placa-board-400-puntos-bp001}{Breadboard}	     & 2                                        & 3,25                                                    & 6,50                                      \\
\rowcolor{lightgray}
\href{https://televoz.org/4955_105led3rojo-ab_diodo-led-3-mm-rojo-alto-brillo}{Light-emitting diodes (LEDs)}                 & 5                                        &  0,35                                                  &1,75          \\
\href{https://televoz.org/5551_001R_RESISTENCIA}{Resistor}                 & 5                                        &  0,15                                                  &1,75 \\
\rowcolor{lightgray}
\href{https://televoz.org/4861_104boardcab20_cable-para-board-20-cms-macho-macho-10-unidades}{Precut wires} & 2 & 2,75 & 5,50 \\
\href{https://store.arduino.cc/arduino-ethernet-shield-2}{Ethernet Shield Arduino} & 1 & 21,50 & 21,50 \\
\rowcolor{lightgray}
\href{https://televoz.org/conexion-utp-cat6-5-00-mts-tr4000029}{UTP Cable} & 1 & 2,5 & 2,5 \\
\href{https://www.seeedstudio.com/Ultra-Sonic-range-measurement-module-p-626.html}{Ultrasonic sensor module} & 1 & 15 & 15 \\
\rowcolor{lightgray}
\href{https://www.parallax.com/product/pir-sensor-with-led-signal/}{PIR sensor} & 1 & 14,99 & 14,99 \\
\textbf{Total} & & & 109,93 \\
\end{tabular}
\end{table}

\begin{table}[H]
\caption{Raspberry Pi server}
\begin{tabular}{llll}
\hline
	& \textbf{Quantity}& \textbf{Price per unit (euro)}& \textbf{Total} \\ \hline
	\rowcolor{lightgray}
\href{https://www.elcorteingles.es/electronica/A38457605-tarjeta-de-memoria-samsung-microsd-evo-plus-mb-mc64ha-64gb/#}{MicroSD card 64GB} & 1 & 15,99 & 15,99 \\
\href{https://www.tiendatec.es/raspberry-pi/gama-raspberry-pi/1099-raspberry-pi-4-modelo-b-2gb-765756931175.html?src=raspberrypi}{Raspberry Pi 4 2GB}   & 1                                        & 44,95                                              & 44,95                                 \\
\rowcolor{lightgray}
\href{https://televoz.org/conexion-usb-tipo-a-macho-a-usb-c-1-00-mts}{USB A to USB C connection} & 1 & 3,50 & 3,50 \\
\href{https://televoz.org/694_888con170n_adaptador-hdmi-hembra-a-micro-hdmi-macho-con170n}{HDMI to microHDMI adapter} & 1 & 2,55 & 2,55 \\
\rowcolor{lightgray}
\href{https://televoz.org/conexion-usb-tipo-a-macho-a-usb-c-1-00-mts}{USB A to USB C connection} & 1 & 3,50 & 3,50 \\
\href{https://televoz.org/9230_992tv4000046_conexion-hdmi-macho-macho-1-00mts-tv40-00046}{HDMI connector} & 1 & 2,5 & 2,5 \\
\rowcolor{lightgray}
\href{https://televoz.org/teclado-usb-aqprox-appkbeco}{USB Keyboard} & 1 & 6,95 & 6,95 \\
\textbf{Total} & & & 79,94 \\
\end{tabular}
\end{table}

\section{Hourly distribution of workload}
Some charts below will give a visual distribution of the workload for the realization of this project along the time. There will be as many tables as chapters are in this projectThe number of total hours invested are 360 with an error margin that will be assumed as negligible.

\begin{table}[H]
\caption{Prior study chart}
\begin{tabular}{llll}
\hline
                               \textbf{Task}    & \textbf{Estimated w. hours} & \textbf{Actual w. hours} \\ \hline
\rowcolor{lightgray}
Search of IT projects related to farming        & 10                          & 20                     \\
Make projects' selection                        & 1                           & 2                      \\
\rowcolor{lightgray}
Create structure of document                    & 2                           & 3                      \\
Report writing                                  & 5                           & 10                     \\
\rowcolor{lightgray}
\textbf{Global computation}                     & 18                          & 35                     \\
\end{tabular}
\end{table}

\begin{table}[H]
\caption{farmOS chart}
\begin{tabular}{llll}
\hline
                               \textbf{Task}    & \textbf{Estimated w. hours} & \textbf{Actual w. hours} \\ \hline
\rowcolor{lightgray}
Research about farmOS project                   & 10                          & 15                     \\
Installation from scratch                       & 1                           & 5                      \\
\rowcolor{lightgray}
Make some tests and screenshots                 & 3                           & 5                      \\
Create structure of document                    & 1                           & 1                      \\
\rowcolor{lightgray}
Report writing                                  & 5                           & 7                     \\
\textbf{Global computation}                     & 20                          & 33                     \\
\end{tabular}
\end{table}

\begin{table}[H]
\caption{Budget chart}
\begin{tabular}{llll}
\hline
                               \textbf{Task}    & \textbf{Estimated w. hours} & \textbf{Actual w. hours} \\ \hline
\rowcolor{lightgray}
Create structure of document                    & 1                           & 5                      \\
Report writing                                  & 1                           & 1                      \\
\rowcolor{lightgray}
Create hardware tables                          & 1                           & 3                     \\
Create workload tables                          & 1                           & 4                      \\
\rowcolor{lightgray}
\textbf{Global computation}                     & 4                          & 13                     \\
\end{tabular}
\end{table}

\begin{table}[H]
\caption{Arduino implementation}
\begin{tabular}{llll}
\hline
                               \textbf{Task}    & \textbf{Estimated w. hours} & \textbf{Actual w. hours} \\ \hline
\rowcolor{lightgray}
Create structure of document                    & 1                           & 5                      \\
Report writing                                  & 1                           & 1                      \\
\rowcolor{lightgray}
Create hardware tables                          & 1                           & 3                     \\
Create workload tables                          & 1                           & 4                      \\
\rowcolor{lightgray}
\textbf{Global computation}                     & 4                          & 13                     \\
\end{tabular}
\end{table}

\begin{table}[H]
\caption{nombre de tablita}
\begin{tabular}{llll}
\hline
                               \textbf{Task}   & \textbf{Hours} \\ \hline
\rowcolor{lightgray}                                           
Prior study                                     & 60           \\
Structure of document                           & 1 \\
\rowcolor{lightgray}                                           
Read about Ubuntu install with ddev in Raspi    & 1           \\
Sensor especification                           & 10           \\
\rowcolor{lightgray}                                           
Raspberry Pi install and config                 & 1           \\
Write tests on Arduino sensor implementation    & - \\
\rowcolor{lightgray}                                           
Write tests on Raspberry install and config     & - \\  
Make chart for budget                           & - \\
\rowcolor{lightgray}                                           
Sensor implementation                           & - \\
Tests on sensors: checklists                    & - \\
\rowcolor{lightgray}                                           
Ethernet shield implementation                  & - \\
Remote communication implementation             & - \\
\rowcolor{lightgray}                                           
farmOS installation                             & - \\
Contribution to farmOS                          & -
\end{tabular}
\end{table}

\section{Conclusions}
One of the main conclusions that can be drawn from this section is that the main issue of this prototype is not merely economic, as might be thought, but focuses in offering a different way of managing a farm (or any other harvesting) that does not compromise the privacy and security of the user.

Because there is no third party company processing the data and because the software used is licensed as GPLv2, it can be assured and guaranteed that the treatment of the data is always with the consent and knowledge of the user and owner.
